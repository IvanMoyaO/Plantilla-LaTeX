% ============
% IDIOMA
% ============
\usepackage[utf8]{inputenc}  % usar utf8
\usepackage[spanish]{babel}
\decimalpoint % Usar punto decimal

% ============
% ESPACIADO
% ============
\usepackage{setspace} 
\setstretch{1}
% sangría
\usepackage{indentfirst}
\setlength{\parskip}{0.2cm}       
\setlength{\parindent}{0.5cm}


% ============
% MATEMÁTICAS
% ============
\usepackage{amsmath}
\usepackage{amssymb}
\usepackage{mathtools}

% ============
% APARIENCIA
% ============
\usepackage{fontawesome5} % Simbolitos (es font-awesome 5)
\usepackage{bigints} % integrales bonitas
\usepackage{xcolor} % colores
\usepackage{graphicx} 
\usepackage[most]{tcolorbox} % cajitas
\usepackage{fancyhdr} % ver page_style (\pagestyle{fancy})

\usepackage[paper=A4]{typearea} % Márgenes
\usepackage[top=3cm, bottom=3cm,
            left = 2cm, right = 2cm,
            headsep = 1cm, 
            ignoremp]{geometry}
            
\usepackage[Bjornstrup]{fncychap} % Capítulos

% ============
% MISC
% ============
\usepackage{hyperref} % índice clickable

